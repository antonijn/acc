\documentclass[12pt, a4paper]{article}
\usepackage{amsmath, listingsutf8, mathtools}
\title{Exploring SSA Optimization Techniques}
\author{A. D. Blom}
\date{}
\begin{document}
  \maketitle
  \begin{abstract}
In this era of global digitalization, the need for performant software is larger
than ever. For the production of fast software not only 
fast hardware and solidly written code are needed, but also a well-optimizing 
compiler. This paper explores some common optimization techniques based on an 
SSA intermediate language.
  \end{abstract}

  \section{Introduction}

  \section{Intermediate representation used}
The intermediate representation (IR) used here is an SSA implentation, 
partially based on the LLVM IR. There are a few differences: the IR used here 
uses a typesystem analogous to C's, and uses a C-like syntax for textual 
representation as well.

The custom IR used also lacks implicit blocks, so blocks are declared 
explicitly using labels, and follow temporary value numbering.

Concretlely, the following LLVM snippet:

\begin{lstlisting}
	%1 = add i32 1, i32 2
	%2 = icmp eq i32 %1, i32 3
	br i1 %2, label %3, label %0
	ret i32 0
\end{lstlisting}

Is equivalent to the following custom-IR:

\begin{lstlisting}
L0:	int t1 = add((int)1, (int)2);
	_Bool t2 = cmp eq(t1, (int)3);
	split(t2, L3, L0);
L3:	ret((int)0);
\end{lstlisting}

Both languages feature an \verb+undef+ constant, with the ability to take on any
value. The custom IR lacks a \verb+null+ constant, it uses \verb+0+ instead.

A working C front-end is expected to be present, and snippets here will be 
either in custom IR or C.


  \section{Phiability optimization}
  \subsection{Problem}
Imperative languages usually allow variables and memory to be rewritten. SSA 
inherently, does not allow temporary variables to change their value over time. 
SSA supports several ways to support this model. One is using the \verb+alloca+
instruction to allocate memory and use the \verb+load+/\verb+store+ system,
another, is cleverly using $\phi$ (phi) nodes.

Consider the following imperative code:

\begin{lstlisting}
	int i;
	if (condition)
		i = 0;
	else
		i = 1;
	return i;
\end{lstlisting}

A naive translation would be:

\begin{lstlisting}
L0:	int* t1 = alloca(int);
	split(cond, L2, L3);
L2:	store(t1, (int)0);
	jmp(L4);
L3:	store(t1, (int)1);
	jmp(L4);
L4:	int t5 = load(t1);
	ret(t5);
\end{lstlisting}

It involves two memory accesses and a memory allocation. Pointers are involved
so it's hard to optimize any further.

It also complicates register allocation a great deal. The allocator now not 
only needs to keep track of where its temporaries are, but also the registers 
used by the \verb+alloca+ instructions. Furthermore it's complicated by 
requiring a new lifetime analysis method, instead of the one already provided 
for temporaries, since most \verb+alloca+ memory needn't be alive for the 
entirety of the
surrounding function.

A system that would allow turning this \verb+alloca+ system into a more
SSA-appropriate  system would get rid of all these cases where alloca would form
an exception, by morphing an advanced \verb+load+/\verb+store+ mechanism into a
mechanism using mostly temporaries for storage.

SSA features a mechanism that allows selecting a value based on the previously 
run block. This system is a $\phi$ node system, where a $\phi$ node is an
instruction taking a map of blocks and expressions, selecting the appropriate
expression based on the predecessing block.

\begin{lstlisting}
L0:	split(cond, L1, L2);
L1:	jmp(L3);
L2:	jmp(L3);
L3:	int t4 = phi(	L1, (int)0,
			L2, (int)1	);
	ret(t4);
\end{lstlisting}

Sometimes for imperative languages it is impossible to use the second system, 
for example in the case where actual memory is required:

\begin{lstlisting}
	int i = 0;
	foo(&i);
	return i;
\end{lstlisting}

Can't use a $\phi$ node system, since it needs \verb+i+ to actually exist in 
memory. It is therefore hard for a front-end to decide which system to use, and 
many default to using the first system all of the time, relying on the 
middle-end to optimize it into a $\phi$ node system. If an \verb+alloca+ 
variable can convert its \verb+load+/\verb+store+ system it shall be considered 
\textit{phiable}.

\subsection{Implementation}
Given a simple, one-block SSA graph:

\begin{lstlisting}
L0:	int* t1 = alloca(int);
	store(t1, (int)0);
	int t2 = load(t1);
	int t3 = add(t1, (int)10);
	store(t1, t3);
	int t4 = load(t1);
	ret(t4);
\end{lstlisting}

Can \verb+t1+ be considered phiable? It's been established that an alloca system 
is not phiable if the memory is actually required to exist. This means (naively) 
that a system is not phiable if the \verb+alloca+ instruction is used outside 
its own \verb+load+/\verb+store+ instructions: that is if it is ever used in an 
instruction, except as the first operand of a \verb+store+ or \verb+load+.

\verb+t1+ meets the phiability requirements. \verb+load+ instructions need to be 
replaced by its last \verb+store+. This means that the \verb+load+ in line 3 
(\verb+t2+) needs to be replaced by its last stored value (line 2). \verb+t4+ 
similarly needs to be replaced by \verb+t3+:

\begin{lstlisting}
L0:	int t1 = add((int)0, (int)10);
	ret(t1);
\end{lstlisting}

This constitutes an enormous code shrinkage, and will speed up the code 
immensely.

Finding the last store is trivial for these one-block examples, it is more 
involved when considering a piece of code where the last store is in one of a 
\verb+load+'s block predecessors. Consider this:

\begin{lstlisting}
L0:	int* t1 = alloca(int);
	split(cond, L2, L3);
L2:	store(t1, (int)0);
	jmp(L4);
L3:	store(t1, (int)1);
	jmp(L4);
L4:	int t5 = load(t1);
	ret(t5);
\end{lstlisting}

For the load in line 7 for example, finding the last store is non-trivial, it 
has in fact got multiple last store instructions, one in \verb+L1+ and one in 
\verb+L2+. It is now actually required to implement a  $\phi$ node. It selects 
the value from L1 if that was its predecessor, and the store from \verb+L2+ if 
that was its predecessor using a $\phi$ node:

\begin{lstlisting}
L0:	split(cond, L1, L2);
L1:	jmp(L3);
L2:	jmp(L3);
L3:	int t4 = phi(L1, (int)0, L2, (int)1);
	ret(t4);
\end{lstlisting}

It is also possible for an \verb+alloca+ to be loaded without any previous 
\verb+store+. In that case, the value of the \verb+load+ is undefined, and it is 
tempting to use the \verb+undef+ constant. It is important, however, that the 
result of the load is guaranteed to remain constant. That isn't the case if all 
instances are replaced by individual \verb+undef+ constants. Consider, for 
instance, the following example:

\begin{lstlisting}
L0:	int* t1 = alloca(int);
	int t2 = load(t1);
	int t3 = load(t1);
	_Bool t4 = cmp eq(t2, t3);
	...
\end{lstlisting}

The value of \verb+t4+ is well-defined, because the value of \verb+t1+ is
guaranteed not to alter spontaneously. If the following translation would be 
used:

\begin{lstlisting}
L0:	_Bool t1 = cmp eq((int)undef, (int)undef);
	...
\end{lstlisting}

The result of the comparison is undefined as well.

It is therefore required to introduce a \verb+undef+ instruction. The code would 
therefore be
optimized into:

\begin{lstlisting}
L0:	int t1 = undef(int);
	_Bool t4 = cmp eq(t1, t1);
	...
\end{lstlisting}


\section{Constant folding}
\subsection{Problem}
When a programmer writes something along these lines:

\begin{lstlisting}
	int i = 10 - 3 * 2;
\end{lstlisting}

The compiler can be expected to see that i should be initialised to four, rather 
than having it emit instructions for each mathematical operation. Moreover, if a 
programmer types:

\begin{lstlisting}
	int a = 10;
	int b = a * 2;
\end{lstlisting}

The compiler can also be expected to simplify the initialisation of b into an 
initialisation to twenty. Although perhaps trivially optimised manually, these
types of trivial constant expressions occur not so much in manually written code,
but quite often in macro expansions.

Therefore the compiler may not expect all constants to be simplified as much as
possible. Instead, the compiler evaluates these constants in a process known as 
constant folding, and subsequently propegates these constants further, filling 
them in for SSA variables along the way in a process known as constant 
propagation.

\subsection{Implementation}
In order to perform any useful consant folding, the compiler needs to fill in 
constants for variables where possible, so code of the form:

\begin{lstlisting}
	int a = 10;
	int b = a * a;
	return b - a;
\end{lstlisting}

Becomes:

\begin{lstlisting}
	int b = 10 * 10;
	return b - 10;
\end{lstlisting}

Once the value of b is determined, it should then also be filled in, to fold 
further. Since the value of b is 100, it can be used to fill in the return 
expression:

\begin{lstlisting}
	return 100 - 10;
\end{lstlisting}

This value can then be folded once more to yield the value 90:

\begin{lstlisting}
	return 90;
\end{lstlisting}

This algorithm might look quite involved, but its simplicity is actually 
staggering. It simply depends on phiability optimisation. Phiability 
optimisation fills in constants for variables automatically. Consider the first 
fragment's IR before phiability optimisation:

\begin{lstlisting}
L0:	int* t1 = alloca(int);
	int* t2 = alloca(int);
	store((int)10, t1);
	int t3 = load(t1);
	int t4 = load(t1);
	int t5 = mul(t3, t4);
	store(t5, t2);
	int t6 = load(t2);
	int t7 = load(t1);
	int t8 = sub(t6, t7);
	ret(t8);
\end{lstlisting}

The variables still exist in their crude memory form. However, their values are 
propagated automatically once phiability optimisation occurs:

\begin{lstlisting}
L0:	int t1 = mul((int)10, (int)10);
	int t2 = sub(t1, (int)10);
	ret(t2);
\end{lstlisting}

The constants can now be propagated with a pass that scans for computable 
instructions (arithmetic instructions of which both operands are constants) and 
computes their values, filling them in for all future occurrences:

\begin{lstlisting}
L0:	ret((int)90);
\end{lstlisting}

\subsection{Considerations}
\subsubsection{Platform incompatibilities}
There is a way compiler-based constant folding might stand in the way of the 
programmer. Mostly the compiler can do this when folding away instructions 
operating on floating point operands, because different targets may compute 
floating point operations differently. Therefore cross-compilation becomes an 
issue; if a floating point instruction for target Y normally yielding Vy, it 
yields Vx when folded away by target X, causing different semantics before and 
after optimisation.

A solution to this problem is to implement a floating point virtual machine for 
several targets, that use non IEEE floating point. Targets using IEEE floating 
point can use C99's internal way of computing IEEE floating point operations.
Since implementing such a system is non-trivial, code duplication needs to be 
avoided. If any other optimisation would need to be able to calculate an 
operation on two constants, it should run the same code. Therefore, the actual 
folding computations are performed outside of the optimiser, by a separate 
folding system.

\end{document}
